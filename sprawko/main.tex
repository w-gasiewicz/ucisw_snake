\documentclass[12pt]{article}
\usepackage{polski}
\usepackage[utf8]{inputenc}
\usepackage{amsmath}
\usepackage{mathtools}
\usepackage{hyperref}
\usepackage{tikz}
\tikzset{
  heap/.style={
    every node/.style={shape=rectangle,rounded corners, draw},
    level 1/.style={sibling distance=30mm},
    level 2/.style={sibling distance=10mm}
  }
}
\usepackage{blkarray, bigstrut}
\usepackage{graphicx}
\usepackage[colorinlistoftodos]{todonotes}
\usepackage[left=2cm,top=3cm,right=2cm,bottom=3cm,bindingoffset=0.5cm]{geometry}
\usepackage{multicol}
\usepackage{afterpage}
\usepackage{array}
\usepackage{pgfplots}
\usepackage{titlesec}
\usepackage{listings}
\lstset{
	showstringspaces=false
  basicstyle=\ttfamily,
  columns=fullflexible,
  frame=single,
  breaklines=true,
  postbreak=\mbox{\textcolor{red}{$\hookrightarrow$}\space},
  escapeinside={(*@}{@*)},
}
\newcommand{\horrule}[1]{\rule{\linewidth}{#1}} % Create horizontal rule command with 1 argument of height
\usepackage{fancyhdr} % Headers and footers
\usepackage{tabularx}
\usepackage{listings}
\usepackage{dirtree}
\usepackage{caption}
\usepackage[section]{placeins}
\usepackage{amsfonts}
\captionsetup[figure]{font=small,labelfont=small}
\captionsetup[table]{font=small,labelfont=small}

\newcommand\blankpage{%
    \null
    \thispagestyle{empty}%
    \addtocounter{page}{-1}%
    \newpage}
    
\pgfplotsset{compat=1.16}
\setcounter{tocdepth}{3}
\setcounter{secnumdepth}{3}
\setlength{\marginparwidth}{2cm}
 
\titleformat
{\section} % command
[display] % shape
{\bfseries\LARGE} % format
{} % label
{0.5ex} % sep
{
} % before-code
[
\vspace{-0.5ex}%
\rule{\textwidth}{0.3pt}
] % after-code

\begin{document}
\begin{titlepage}

\newcommand{\HRule}{\rule{\linewidth}{0.5mm}} % Defines a new command for the horizontal lines, change thickness here

\center % Center everything on the page
\vspace*{\fill}
%----------------------------------------------------------------------------------------
%   HEADING SECTIONS
%----------------------------------------------------------------------------------------

\textsc{\LARGE Politechnika Wrocławska}\\[0.5cm] % Name of your university/college
\textsc{\Large Wydział Elektroniki}\\[1.5cm] % Major heading such as course name
\textsc{\large Układy Cyfrowe i Systemy Wbudowane 2}\\[0.5cm] % Minor heading such as course title

%----------------------------------------------------------------------------------------
%   TITLE SECTION
%----------------------------------------------------------------------------------------

\HRule \\[0.1cm]
{ \Huge Raport z projektu\\\vspace{10pt}Gra snake}\\[0.1cm] % Title of your document
\HRule \\[1.5cm]
 
%----------------------------------------------------------------------------------------
%   AUTHOR SECTION
%----------------------------------------------------------------------------------------

\begin{minipage}{0.3\textwidth}
\begin{flushleft} \large
\textit{Autorzy:}\\
Krystian Plata\\
Wojciech Gąsiewicz\\
\end{flushleft}
\end{minipage}
\begin{minipage}{0.2\textwidth}
\begin{flushleft}\large
\vspace{\baselineskip}
236250\\
235086\\
\end{flushleft}
\end{minipage}
\begin{minipage}{0.4\textwidth}
\begin{flushright} \large
\emph{Prowadzący:} \\
Dr inż. Dariusz Banasiak\\ % Supervisor's Name
\end{flushright}
\end{minipage}\\

% If you don't want a supervisor, uncomment the two lines below and remove the section above
%\Large \emph{Author:}\\
%John \textsc{Smith}\\[3cm] % Your name

%----------------------------------------------------------------------------------------
%   DATE SECTION
%----------------------------------------------------------------------------------------
\vspace*{\fill}


{\large \today}\\[2cm] % Date, change the \today to a set date if you want to be precise

% \vfill % Fill the rest of the page with whitespace
\end{titlepage}
\newpage
\tableofcontents
\clearpage
\section{Wstęp}
\subsection{Cel projektu}
Celem projektu było zaimplementowanie popularnej gry snake przy wykorzystaniu 
układu Spartan-3E. Wizualizacja została zrealizowana przy użyciu monitora VGA. 
Do strowania grą została wykorzystana dostępna w laboratorium klawiatura wyposażona w 
złącze PS2. W grze zotały wykorzystane moduły dostępne na stronie laboratorium oraz własne.
\subsection{Zasady gry}
Podczas rozgrywki użytkownik decyduje o tym w którym kierunku uda się wąż. 
Sterowanie odbywa się poprzez klawisze W,A,S,D.
\begin{itemize}
\item W - ruch w górę.
\item A - ruch w lewo.
\item S - ruch w dół.
\item D - ruch w prawo.
\end{itemize}
Ruchy są ograniczone tak, że niemożliwe jest cofnięcie się w tej samej osi ruchu.
Wąż przesuwa się zgodnie z aktualnym torem o jedną jednostkę, co ustalony okres czasu.
W trakcie działania gry, generowana jest losowa pozycja pożywienia. 
Zjedzenie pożywienia przez węża skutkuje wzrostem jego długości o jedną jednostkę.
Warunkiem końca gry jest uderzenie wężem w ramkę na ekranie.
\subsection{Zadania i harmonogram}
W początkowym etapie projektowania określono zadania, które należało wykonać 
na odpowiednich zajęciach:
\begin{enumerate}
    \item Obsługa monitora - implementacja modułu obsługującego sygnały ze standardu VGA.
    \item Implementacja modułu odpowiedzialnego za obsługę klawiatury.
    \item Fizyka węża - implementacja modułu, który powoduje ruch węża zgodny z regułami. 
    \item Implementacja warunków ograniczających.
    \item Implementacja procesów odpowiedzialnych za generowanie pokarmu.
    \item Implementacja procesów odpowiedzialnych za wydłużanie węża po zjedzeniu pokarmu.
\end{enumerate}
\subsection{Wykorzystany sprzęt oraz technologie}
Do implementacji gry wykorzystany został język opisu sprzętu VHDL. Stanowiska laboratoryjne, 
na których przebiegała implementacja były wyposażone w oprogramowanie Xilinx ISE. 
Środowisko to było niezbędne podczas kompilacji programu pod sprzęt wykorzystany 
do jego obsługi (Xilinx Spartan 3E Starter). 
\subsection{Spis modułów}
\textbf{Moduły gotowe:}
\begin{itemize}
    \item PS2-Kbd – Obsługuje komunikację z klawiaturą podpiętą przez złącze PS2.
\end{itemize}
\textbf{Moduły autorskie:}
\begin{itemize}
    \item \textbf{VGADriver} – Przechowuje pozycję aktualnego piksela zgodnie z ustaloną 
    synchronizacją.
    \item \textbf{KbdDir} – Zamienia informację wejściową w postaci kodu klawisza,
     na informację o kierunku węża.
    \item \textbf{Snake} – Przechowuje aktualne współrzędne snake, zmienia je z 
    określoną częstotliwością oraz zwraca informację, o kolorze przetwarzanego piksela.
\end{itemize}
\subsection{Hierarchia plików źródłowych}
Głównym plikiem projektu, na podstawie którego była przeprowadzana synteza był schemat modułów.
Hierarchię plików w początkowej fazie projektu zamieszczono poniżej:
\dirtree{%
.1 snake\_schematic.sch.
.2 VGADriver.vhd.
.2 KbdDir.vhd.
.2 Snake.vhd.
.2 GenIO.ucf.
.2 PS2Kbd.ngc.
}
\clearpage
\section{Schemat projektu}
\newpage
\section{Moduł PS2\_Kbd}
\section{Moduł VGADriver}
\section{Moduł KbdDir}
\section{Moduł Snake}
\section{Wnioski}
\end{document}
